%%%%%%%%%%%%%%%%%%%%%%%%%%%%%%%%%%%%%%%%%%%%%%%%%%%%%%%%%%%%%%%%%%
%%%%%%%% ICML 2015 EXAMPLE LATEX SUBMISSION FILE %%%%%%%%%%%%%%%%%
%%%%%%%%%%%%%%%%%%%%%%%%%%%%%%%%%%%%%%%%%%%%%%%%%%%%%%%%%%%%%%%%%%

% Use the following line _only_ if you're still using LaTeX 2.09.
%\documentstyle[icml2015,epsf,natbib]{article}
% If you rely on Latex2e packages, like most moden people use this:
\documentclass{article}

% use Times
\usepackage{times}
% For figures
\usepackage{graphicx} % more modern
%\usepackage{epsfig} % less modern
\usepackage{subfigure} 
\usepackage{amssymb}
% For citations
\usepackage{natbib}
\usepackage{amsthm}
\usepackage{amsmath}
\usepackage{amssymb}
\usepackage{relsize}

\newcommand\independent{\protect\mathpalette{\protect\independenT}{\perp}}
\def\independenT#1#2{\mathrel{\rlap{$#1#2$}\mkern2mu{#1#2}}}
\newtheorem{definition}{Definition}
\newtheorem{lemma}{Lemma}
\newtheorem{theorem}{Theorem}
\newtheorem{corollary}{Corollary}
\newenvironment{claimproof}[1]{\par\noindent\underline{Proof:}\space#1}{\hfill $\blacksquare$}

% For algorithms
\usepackage{algorithm}
\usepackage{algorithmic}

% As of 2011, we use the hyperref package to produce hyperlinks in the
% resulting PDF.  If this breaks your system, please commend out the
% following usepackage line and replace \usepackage{icml2015} with
% \usepackage[nohyperref]{icml2015} above.
\usepackage{hyperref}

% Packages hyperref and algorithmic misbehave sometimes.  We can fix
% this with the following command.
\newcommand{\theHalgorithm}{\arabic{algorithm}}

% Employ the following version of the ``usepackage'' statement for
% submitting the draft version of the paper for review.  This will set
% the note in the first column to ``Under review.  Do not distribute.''
\usepackage{icml2015} 

% Employ this version of the ``usepackage'' statement after the paper has
% been accepted, when creating the final version.  This will set the
% note in the first column to ``Proceedings of the...''
%\usepackage[accepted]{icml2015}


% The \icmltitle you define below is probably too long as a header.
% Therefore, a short form for the running title is supplied here:
\icmltitlerunning{Submission and Formatting Instructions for ICML 2015}

\begin{document} 

\twocolumn[
\icmltitle{A Kernel Test for Three-Variable Interactions with Random Processes (ICML 2016)}

% It is OKAY to include author information, even for blind
% submissions: the style file will automatically remove it for you
% unless you've provided the [accepted] option to the icml2015
% package.
\icmlauthor{Your Name}{email@yourdomain.edu}
\icmladdress{Your Fantastic Institute,
            314159 Pi St., Palo Alto, CA 94306 USA}
\icmlauthor{Your CoAuthor's Name}{email@coauthordomain.edu}
\icmladdress{Their Fantastic Institute,
            27182 Exp St., Toronto, ON M6H 2T1 CANADA}

% You may provide any keywords that you 
% find helpful for describing your paper; these are used to populate 
% the "keywords" metadata in the PDF but will not be shown in the document
\icmlkeywords{boring formatting information, machine learning, ICML}

\vskip 0.3in
]

\begin{abstract} 
Explain what this is all about, and the main contributions:

\begin{itemize}
\item Applied Wild Bootstrap to Lancaster test statistic
\item Main theoretical challenge was to show that the conditions required to apply WB are satisfied by Lancaster
\item This was done in a novel way - rather than using the Hoeffding decomposition, we come up with a new method which is simpler, (but requires an extra condition on the timeseries?)
\item We also show that the power of the Lancaster test described in Arthur's original paper can be improved - we show that they used conservative p-values
\end{itemize}
\end{abstract} 

\section{Introduction}
\label{introduction}

\begin{itemize}
\item Describe three variable interaction. It is particularly useful for cases in which any pairwise interaction is weak, but that the three variables interact strongly together.
\item Test consists of two parts - calculating the test statistic, and bootstrapping the statistic to sample from the null in order to calculate the p-value threshold.
\item When using time series, the difficult part is the bootstrapping because shuffling the indices breaks the temporal dependence structure.
\item In [Leucht], they give a method for bootstrapping a certain class of statistics.
\item The main contributions of this paper are the folllowing:
\begin{itemize}
\item To show that the Lancaster test statistic is such a statistic
\item This is done using a new style of technique which in particular gives a significantly simpler proof that HSIC is also such a statistic (and thus simplifies the proofs used in [HSIC+time series])
\item To show that the multiple testing corrections used in [Lancaster] are too conservative, and therefore that we can improve test power by using a more relaxed correction.
\end{itemize}
\end{itemize}


This work combines the works of [HSIC + time series] and [Lancaster interaction] to give a non-parametric test for three variable interactions in which the samples are drawn from random processes.

\section{Background} 

In this section we briefly introduce the theory and definitions required to understand the statement and proof of our main result.

\subsection{Kernel Mean Embedding (and HSIC?)}

Given an integrally strictly positive definite kernel $k$ on a set $\mathcal{Z}$, the mapping induced by $k$ from $\mathcal{M(Z)}$, the set of signed measures on $\mathcal{Z}$, to the RKHS $\mathcal{H}_k$ of $k$ via $m \mapsto \int k(x,\cdot)dm(x)$ is injective. Given a finite sample $z_1,\ldots,z_n$ drawn from a probability distribution $\mathbb{P}_z$, the mean embedding $\mu_{\mathbb{P}_z}$ can be estimated as $\hat\mu_{\mathbb{P}_z} = \frac{1}{n}\sum_{i=1}^{n}k(z_i,\cdot)$. This idea is exploited in the construction of certain statistical tests including two sample independence testing (HSIC) - in this case, we wish to understand whether $\mathbb{P}_{XY}$ factorises as $\mathbb{P}_{X}\mathbb{P}_{Y}$ based on finite samples $(X_i,Y_i)$ drawn from $\mathbb{P}_{XY}$. We can consider distance between the empirical embeddings of the two measures via $\| \hat\mu_{\mathbb{P}_{XY}}  - \hat\mu_{\mathbb{P}_{X}\mathbb{P}_{Y}}\|^2$. We can then bootstrap this statistic to generate samples of it under the null hypothesis that $\mathbb{P}_{XY} = \mathbb{P}_{X}\mathbb{P}_{Y}$ to calculate a threshold distance over which we would reject the null hypothesis and conclude that the distribution does not factorise

\subsubsection{Notation?}

Maybe put all notation here? Need to define gram matrices, empirically centred gram matrices and population centred gram matrices.

\subsection{Lancaster}

The above ideas of injectively embedding measures into a Hilbert space can be extended from the two variable case to consider properties of three or more variables. The Lancaster statistic on the triple of variables $(X,Y,Z)$ is defined as the signed measure $\Delta_LP = \mathbb{P}_{XYZ} - \mathbb{P}_{XY}\mathbb{P}_{Z} - \mathbb{P}_{XZ}\mathbb{P}_{Y} - \mathbb{P}_{X}\mathbb{P}_{YZ} + 2\mathbb{P}_{X}\mathbb{P}_{Y}\mathbb{P}_{Z}$. It is straightforward to show that if any variable is independent of the other two (equivalently, if the joint distribution $\mathbb{P}_{XYZ}$ factorises into a product of marginals in any way), then $\Delta_LP = 0$. That is, writing $\mathcal{A}_X = \{X \independent (Y,Z)\}$ and similar for $\mathcal{A}_Y$ and $\mathcal{A}_Z$, we have that

\[\mathcal{A}_X \enspace \lor \enspace \mathcal{A}_Y \enspace \lor \enspace \mathcal{A}_Z \Rightarrow \Delta_LP=0 \]

Given a finite sample $(X_i,Y_i,Z_i)_{i=1}^n$, the mean embedding of the Lancaster interaction can be empirically estimated as $\Delta_L\hat{P} = \hat{\mu}_{\mathbb{P}_{XYZ}} - \hat{\mu}_{\mathbb{P}_{XY}\mathbb{P}_{Z}} - \hat{\mu}_{\mathbb{P}_{XZ}\mathbb{P}_{Y}} - \hat{\mu}_{\mathbb{P}_{X}\mathbb{P}_{YZ}} + 2\hat{\mu}_{\mathbb{P}_{X}\mathbb{P}_{Y}\mathbb{P}_{Z}}$. We use the squared RKHS norm of this quantity as a test statistic to test the following hypothesis:

$\mathcal{H}_0: \mathcal{A}_X \enspace \lor \enspace \mathcal{A}_Y \enspace \lor \enspace \mathcal{A}_Z $\\
$\mathcal{H}_1: \mathbb{P}_{XYZ}$ does not factorise in any way

Given kernels $k,l$ and $m$ on $\mathcal{X},\mathcal{Y}$ and $\mathcal{Z}$ respectively, $k\otimes l \otimes m$ defines a kernel on $\mathcal{X}\times \mathcal{Y} \times \mathcal{Z}$. We write $K, L$ and $M$ to denote the gram matrices of each kernel with respect to the observations, where for example $K_{ij} = k(X_i,X_j)$. We further write $\tilde{K}, \tilde{L}$ and $\tilde{M}$ for the empirically centred gram matrices, where for example $\tilde{K}_{ij} = k(X_i,X_j) - \frac{1}{n}\sum_ik(X_i,X_j)  - \frac{1}{n}\sum_jk(X_i,X_j) + \frac{1}{n^2}\sum_{ij}k(X_i,X_j)$. We can then write [Lancaster]

$\|\Delta_L\hat{P}\|_{k\otimes l \otimes m}^2 = \frac{1}{n^2}\left(\tilde{K}\circ\tilde{L}\circ\tilde{M}\right)_{++}$

where $\circ$ is the Hadamard (element-wise) product and $A_{++} = \sum_{ij}A_{ij}$.

The next part of the statistical test is to find threshold values of the statistic beyond which we would reject the null hypothesis. In the case that the observations are drawn \emph{iid}, this can be done using a permutation bootstrap method. Since our null hypothesis is a composite of three `sub-hypotheses', we must test each of them separately. We reject the composite null hypothesis if and only if we reject all three of the components. For more information on the details of the bootstrapping method, see [Lancaster].

\subsection{Time series}
In this paper we are extending the existing Lancaster test from the \emph{iid} case to a case in which our observations are drawn from a random process. There are various formalisations of memory or 'mixing' of a random process; of relevance to this paper are the following two:

\subsubsection{$\tau$-mixing}
\begin{definition}
A process $(X_t)_{t}$ is \emph{$\tau$-mixing} if $\tau(r) \longrightarrow 0$ as $r\longrightarrow \infty$, where

\[\tau(r) = \sup_{l\in \mathbb{N}} \frac{1}{l} \sup_{r\leq i_1 \leq \ldots \leq i_l} \tau(\mathcal{F}_0,(X_{i_1}, \ldots, X_{i_l})) \longrightarrow 0\]

where

\[ \tau(\mathcal{M},X) = \mathbb{E} (\sup_{g \in \Lambda} | \int g(t) \mathbb{P}_{X|\mathcal{M}}(dt) -  \int g(t) \mathbb{P}_{X}(dt) |)\]

\end{definition}

\subsubsection{$\beta$-mixing}


\begin{definition}
A process $(X_t)_{t}$ is \emph{$\beta$-mixing} (also known as \emph{absolutely regular}) if $\beta(m) \longrightarrow 0$ as $m\longrightarrow \infty$, where

\[ \beta(m) = \frac{1}{2} \sup_n \sup \sum_{i=1}^I \sum_{j=1}^J | \mathbb{P}(A_i \cap B_j) - \mathbb{P}(A_i)\mathbb{P}(B_j)| \]

where the second supremum is taken  over all finite partitions $\{A_1,\ldots, A_I \}$ and  $\{B_1,\ldots, B_J\}$ of the sample space such that $A_i \in \mathcal{A}_1^n$ and $B_j \in \mathcal{A}_{n+m}^\infty$ and $\mathcal{A}_b^c = \sigma(X_b,X_{b+1},\ldots,X_{c})$
\end{definition}

The concept of $\beta$-mixing will be invoked when applying a central limit theorem in the next section. We will also need the following lemma:

\begin{lemma}\label{lemma:beta}
Suppose that the process $(X_t,Y_t,Z_t)_t$ is $\beta$-mixing. Then any `sub-process' is also $\beta$-mixing (for example $(X_t,Y_t)_t$ or $(X_t)_t$)
\end{lemma}

\subsection{V-statistics}
A V-statistic of a k-argument, symmetric function $f$ given $iid$ observations $\mathcal{S}_n = \{S_1,\ldots,S_n\}$ where each $S_i \sim \mathbb{P}$ is written

\[ V(f,\mathcal{S}) =  \frac{1}{n^k} \mathlarger{\sum}_{1\leq i_1,\ldots, i_k \leq n} f(S_{i_1},\ldots,S_{i_k})\]

In this case, $V(f,\mathcal{S})$ is a biased (but asymptotically unbiased) estimator of $\mathbb{E}_{S_{i_1},\ldots S_{i_k} \sim \mathbb{P}}[f(S_{i_1},\ldots,S_{i_k})]$

In this paper we are only concerned with V-statistics for which $k=2$. We call $nV(f,\mathcal{S})$ \emph{normalised}. We call $f$ the \emph{core} of $V$ and we say that $f$ is \emph{degenerate} if, for any $s_1$, $\mathbb{E}_{S_2 \sim \mathbb{P}}[f(s_1,S_2)] = 0$ in which case we say that $V$ is a \emph{degenerate V-statistic} 

Of relevance to us is the fact that many kernel test statistics can be viewed as normalised V-statistics which, under the null hypothesis, are degenerate. If moreover the test statistics diverge under the alternative hypothesis, the test would be consistent. Our main result is to prove that the Lancaster statistic is asymptotically a degenerate V-statistic.

\subsection{Wild Bootstrap}

I think I should define the wild bootstrap here, so that in the next section I can focus on just proving that it can be applied.


\subsection{Hilbert spaced random variable CLT}

Should we actually state the theorem here? We should include the proof that our situation satisfies the conditions of the theorem regardless though, but maybe in the supplementary section.


\begin{itemize}
\item Kernel mean embedding
\item Lancaster
\item Time series
\begin{itemize}
\item $\tau$-mixing
\item $\beta$-mixing
\item Lemma that sub-processes of $\beta$-mixing processes are $\beta$-mixing
\end{itemize}
\item V-statistics
\item Hilbert space valued random variable central limit theorem
\end{itemize} 

\section{Lancaster Interaction for Random Processes}

(Following kacper's paper...)

In this section we construct the Lancaster Interaction test for random processes. The major difficulty in doing so is showing that the test statistic asymptotically satisfies the conditions of the Wild Bootstrap under the null hypothesis of the test, and therefore the Wild Bootstrap can be used to resample the test statistic and provide consistent thresholds for desired p-values.

The approach taken in this paper can also be applied to the HSIC test statistic to give a simpler proof that the Wild Bootstrap can be used for HSIC+timeseries than that given in [Kacper].

\begin{theorem}
Suppose that $\mathbb{P}_{XYZ} = \mathbb{P}_{XY}\mathbb{P}_Z$ and that $(X_i,Y_i,Z_i)_{i=1}^n$ are drawn from a process that is both:

\begin{itemize}
\item $\beta$-mixing with coefficients $\beta(m)$ satisfying $\sum_{m=1}^{\infty}\beta(m)^{\frac{\delta}{2+\delta}}<\infty$
\item $\tau$-mixing with coefficients $\tau(m)$ satisfying $\sum_{m=1}^\infty m^2 \sqrt{\tau(m)} < \infty$
\end{itemize} . Then, as $n\longrightarrow \infty$, 

\[ n\| \Delta_L\hat{P} \|^2 \longrightarrow \frac{1}{n}\left( \overline{\left( \bar{K} \circ \bar{L}\right) }\circ \bar{M} \right) _{++} \]

and this is a normalised degenerate V-statistic.
\end{theorem}

\begin{corollary}
Suppose in addition to the above that $W$ is drawn from a process satisfying the conditions of [Wild Bootstrap]. Then asymptotically,

\[\frac{1}{n}\left(W^\intercal\left( \overline{\left( \bar{K} \circ \bar{L}\right) }\circ \bar{M} \right)W\right) _{++}\]

has the same distribution as $ n\| \Delta_L\hat{P} \|^2$. 
\end{corollary}

We can therefore use this to generate samples of the test statistic $ n\| \Delta_L\hat{P} \|^2$ under the null hypothesis that $\mathcal{A}_Z$. By symmetry, we can therefore use this 


\begin{itemize}
\item Statement of Wild Bootstrap theorem (maybe in background though?)
\item Proof that Lancaster satisfies WB theorem hypothesis
\item ...
\item Multiple testing correction (maybe in next section though?)
\end{itemize}

\section{p-values for Lancaster test}
\begin{itemize}
\item In [Lancaster], they use the Holm-Bonferroni correction. Show here that this isn't actually necessary - that the 'naive' correction works and is therefore more powerful as we use $[\alpha,\alpha,\alpha]$ as the thresholds rather than $[\alpha/3,\alpha/2,\alpha]$ or whatever.	
\end{itemize}


\section{Experiments}
\subsection{Artificial data}
\subsection{Real data}
Maybe check this out for some data? \url{https://stat.duke.edu/~mw/ts_data_sets.html}

\section{Proofs}
Proof of Lemma \ref{lemma:beta}:
\begin{claimproof} Let us consider $(X_t,Y_t)_t$.
Let us call $\beta_{XYZ}(m)$ the coefficients for the process $(X_t,Y_t,Z_t)_t$, and $\beta_{XY}(m)$ the coefficients for the process $(X_t,Y_t)_t$. 

Observe that for $A \in \sigma((X_b,Y_b),\ldots, (X_c,Y_c))$, it is the case that $A \times \mathcal{Z} \in \sigma((X_b,Y_b,Z_b),\ldots, (X_c,Y_c,Z_c))$ and $\mathbb{P}_{XY}(A) = \mathbb{P}_{XYZ}(A\times \mathcal{Z})$.

Thus

\begin{align*}
\beta_{XY}(m) &= \frac{1}{2} \sup_n \sup_{ \{A_i^{XY} \}, \{B_j^{XY} \} } \sum_{i=1}^I \sum_{j=1}^J | \mathbb{P}_{XY}(A_i^{XY} \cap B_j^{XY}) - \mathbb{P}_{XYZ}(A_i^{XY})\mathbb{P}_{XYZ}(B_j^{XY})| \\
&= \frac{1}{2} \sup_n \sup_{ \{A_i^{XY} \}, \{B_j^{XY} \} } \sum_{i=1}^I \sum_{j=1}^J | \mathbb{P}_{XYZ}((A_i^{XY}\times \mathcal{Z}) \cap (B_j^{XY} \times \mathcal{Z})) \\& \quad \quad\quad \quad \quad \quad\quad \quad \quad \quad\quad \quad- \mathbb{P}_{XYZ}(A_i^{XY}\times \mathcal{Z})\mathbb{P}_{XYZ}(B_j^{XY} \times \mathcal{Z})| \\
& \leq \frac{1}{2} \sup_n \sup_{ \{A_i^{XYZ} \}, \{B_j^{XYZ} \} } \sum_{i=1}^I \sum_{j=1}^J | \mathbb{P}_{XYZ}(A_i^{XYZ} \cap B_j^{XYZ}) - \mathbb{P}_{XYZ}(A_i^{XYZ})\mathbb{P}_{XYZ}(B_j^{XYZ})| \\
& = \beta_{XYZ}(m)
\end{align*}

Thus we have shown that  $\beta_{XYZ}(m) \longrightarrow 0 \implies \beta_{XY}(m) \longrightarrow 0$. That is, if  $(X_t,Y_t,Z_t)_t$ is $\beta$-mixing then so is  $(X_t,Y_t)_t$ 

A similar argument holds for any other sub-process.
\end{claimproof}

\section*{Acknowledgments} 
cheers!
\bibliography{example_paper}
\bibliographystyle{icml2015}

\end{document} 


% This document was modified from the file originally made available by
% Pat Langley and Andrea Danyluk for ICML-2K. This version was
% created by Lise Getoor and Tobias Scheffer, it was slightly modified  
% from the 2010 version by Thorsten Joachims & Johannes Fuernkranz, 
% slightly modified from the 2009 version by Kiri Wagstaff and 
% Sam Roweis's 2008 version, which is slightly modified from 
% Prasad Tadepalli's 2007 version which is a lightly 
% changed version of the previous year's version by Andrew Moore, 
% which was in turn edited from those of Kristian Kersting and 
% Codrina Lauth. Alex Smola contributed to the algorithmic style files.  
