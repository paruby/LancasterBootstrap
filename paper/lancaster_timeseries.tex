%%%%%%%%%%%%%%%%%%%%%%%%%%%%%%%%%%%%%%%%%%%%%%%%%%%%%%%%%%%%%%%%%%
%%%%%%%% ICML 2015 EXAMPLE LATEX SUBMISSION FILE %%%%%%%%%%%%%%%%%
%%%%%%%%%%%%%%%%%%%%%%%%%%%%%%%%%%%%%%%%%%%%%%%%%%%%%%%%%%%%%%%%%%

% Use the following line _only_ if you're still using LaTeX 2.09.
%\documentstyle[icml2015,epsf,natbib]{article}
% If you rely on Latex2e packages, like most moden people use this:
\documentclass{article}

% use Times
\usepackage{times}
% For figures
\usepackage{graphicx} % more modern
%\usepackage{epsfig} % less modern
\usepackage{subfigure} 

% For citations
\usepackage{natbib}

% For algorithms
\usepackage{algorithm}
\usepackage{algorithmic}

% As of 2011, we use the hyperref package to produce hyperlinks in the
% resulting PDF.  If this breaks your system, please commend out the
% following usepackage line and replace \usepackage{icml2015} with
% \usepackage[nohyperref]{icml2015} above.
\usepackage{hyperref}

% Packages hyperref and algorithmic misbehave sometimes.  We can fix
% this with the following command.
\newcommand{\theHalgorithm}{\arabic{algorithm}}

% Employ the following version of the ``usepackage'' statement for
% submitting the draft version of the paper for review.  This will set
% the note in the first column to ``Under review.  Do not distribute.''
\usepackage{icml2015} 

% Employ this version of the ``usepackage'' statement after the paper has
% been accepted, when creating the final version.  This will set the
% note in the first column to ``Proceedings of the...''
%\usepackage[accepted]{icml2015}


% The \icmltitle you define below is probably too long as a header.
% Therefore, a short form for the running title is supplied here:
\icmltitlerunning{Submission and Formatting Instructions for ICML 2015}

\begin{document} 

\twocolumn[
\icmltitle{A Kernel Test for Three-Variable Interactions with Random Processes}

% It is OKAY to include author information, even for blind
% submissions: the style file will automatically remove it for you
% unless you've provided the [accepted] option to the icml2015
% package.
\icmlauthor{Your Name}{email@yourdomain.edu}
\icmladdress{Your Fantastic Institute,
            314159 Pi St., Palo Alto, CA 94306 USA}
\icmlauthor{Your CoAuthor's Name}{email@coauthordomain.edu}
\icmladdress{Their Fantastic Institute,
            27182 Exp St., Toronto, ON M6H 2T1 CANADA}

% You may provide any keywords that you 
% find helpful for describing your paper; these are used to populate 
% the "keywords" metadata in the PDF but will not be shown in the document
\icmlkeywords{boring formatting information, machine learning, ICML}

\vskip 0.3in
]

\begin{abstract} 
Explain what this is all about, and the main contributions:

\begin{itemize}
\item Applied Wild Bootstrap to Lancaster test statistic
\item Main theoretical challenge was to show that the conditions required to apply WB are satisfied by Lancaster
\item This was done in a novel way - rather than using the Hoeffding decomposition, we come up with a new method which is simpler, (but requires an extra condition on the timeseries?)
\item We also show that the power of the Lancaster test described in Arthur's original paper can be improved - we show that they used conservative p-values
\end{itemize}
\end{abstract} 

\section{Introduction}
\label{introduction}

\begin{itemize}
\item Describe three variable interaction. It is particularly useful for cases in which any pairwise interaction is weak, but that the three variables interact strongly together.
\item Test consists of two parts - calculating the test statistic, and bootstrapping the statistic to sample from the null in order to calculate the p-value threshold.
\item When using time series, the difficult part is the bootstrapping because shuffling the indices breaks the temporal dependence structure.
\item In [Leucht], they give a method for bootstrapping a certain class of statistics.
\item The main contributions of this paper are the folllowing:
\begin{itemize}
\item To show that the Lancaster test statistic is such a statistic
\item This is done using a new style of technique which in particular gives a significantly simpler proof that HSIC is also such a statistic (and thus simplifies the proofs used in [HSIC+time series])
\item To show that the multiple testing corrections used in [Lancaster] are too conservative, and therefore that we can improve test power by using a more relaxed correction.
\end{itemize}
\end{itemize}


This work combines the works of [HSIC + time series] and [Lancaster interaction] to give a non-parametric test for three variable interactions in which the samples are drawn from random processes.

\section{Background} 

\begin{itemize}
\item Kernel mean embedding
\item Lancaster
\item Time series
\begin{itemize}
\item $\tau$-mixing
\item $\beta$-mixing
\item Lemma that sub-processes of $\beta$-mixing processes are $\beta$-mixing
\end{itemize}
\item V-statistics
\item Hilbert space valued random variable central limit theorem
\end{itemize} 

\section{Lancaster Interaction for Random Processes}
\begin{itemize}
\item Statement of Wild Bootstrap theorem (maybe in background though?)
\item Proof that Lancaster satisfies WB theorem hypothesis
\item ...
\item Multiple testing correction (maybe in next section though?)
\end{itemize}

\section{p-values for Lancaster test}
\begin{itemize}
\item In [Lancaster], they use the Holm-Bonferroni correction. Show here that this isn't actually necessary - that the 'naive' correction works and is therefore more powerful as we use $[\alpha,\alpha,\alpha]$ as the thresholds rather than $[\alpha/3,\alpha/2,\alpha]$ or whatever.	
\end{itemize}


\section{Experiments}
\subsection{Artificial data}
\subsection{Real data}
Maybe check this out for some data? \url{https://stat.duke.edu/~mw/ts_data_sets.html}

\section{Proofs}

\section*{Acknowledgments} 
cheers!
\bibliography{example_paper}
\bibliographystyle{icml2015}

\end{document} 


% This document was modified from the file originally made available by
% Pat Langley and Andrea Danyluk for ICML-2K. This version was
% created by Lise Getoor and Tobias Scheffer, it was slightly modified  
% from the 2010 version by Thorsten Joachims & Johannes Fuernkranz, 
% slightly modified from the 2009 version by Kiri Wagstaff and 
% Sam Roweis's 2008 version, which is slightly modified from 
% Prasad Tadepalli's 2007 version which is a lightly 
% changed version of the previous year's version by Andrew Moore, 
% which was in turn edited from those of Kristian Kersting and 
% Codrina Lauth. Alex Smola contributed to the algorithmic style files.  
